% Options for packages loaded elsewhere
\PassOptionsToPackage{unicode}{hyperref}
\PassOptionsToPackage{hyphens}{url}
%
\documentclass[
]{article}
\usepackage{amsmath,amssymb}
\usepackage{lmodern}
\usepackage{iftex}
\ifPDFTeX
  \usepackage[T1]{fontenc}
  \usepackage[utf8]{inputenc}
  \usepackage{textcomp} % provide euro and other symbols
\else % if luatex or xetex
  \usepackage{unicode-math}
  \defaultfontfeatures{Scale=MatchLowercase}
  \defaultfontfeatures[\rmfamily]{Ligatures=TeX,Scale=1}
\fi
% Use upquote if available, for straight quotes in verbatim environments
\IfFileExists{upquote.sty}{\usepackage{upquote}}{}
\IfFileExists{microtype.sty}{% use microtype if available
  \usepackage[]{microtype}
  \UseMicrotypeSet[protrusion]{basicmath} % disable protrusion for tt fonts
}{}
\makeatletter
\@ifundefined{KOMAClassName}{% if non-KOMA class
  \IfFileExists{parskip.sty}{%
    \usepackage{parskip}
  }{% else
    \setlength{\parindent}{0pt}
    \setlength{\parskip}{6pt plus 2pt minus 1pt}}
}{% if KOMA class
  \KOMAoptions{parskip=half}}
\makeatother
\usepackage{xcolor}
\usepackage[margin=1in]{geometry}
\usepackage{longtable,booktabs,array}
\usepackage{calc} % for calculating minipage widths
% Correct order of tables after \paragraph or \subparagraph
\usepackage{etoolbox}
\makeatletter
\patchcmd\longtable{\par}{\if@noskipsec\mbox{}\fi\par}{}{}
\makeatother
% Allow footnotes in longtable head/foot
\IfFileExists{footnotehyper.sty}{\usepackage{footnotehyper}}{\usepackage{footnote}}
\makesavenoteenv{longtable}
\usepackage{graphicx}
\makeatletter
\def\maxwidth{\ifdim\Gin@nat@width>\linewidth\linewidth\else\Gin@nat@width\fi}
\def\maxheight{\ifdim\Gin@nat@height>\textheight\textheight\else\Gin@nat@height\fi}
\makeatother
% Scale images if necessary, so that they will not overflow the page
% margins by default, and it is still possible to overwrite the defaults
% using explicit options in \includegraphics[width, height, ...]{}
\setkeys{Gin}{width=\maxwidth,height=\maxheight,keepaspectratio}
% Set default figure placement to htbp
\makeatletter
\def\fps@figure{htbp}
\makeatother
\usepackage[normalem]{ulem}
\setlength{\emergencystretch}{3em} % prevent overfull lines
\providecommand{\tightlist}{%
  \setlength{\itemsep}{0pt}\setlength{\parskip}{0pt}}
\setcounter{secnumdepth}{-\maxdimen} % remove section numbering
\ifLuaTeX
  \usepackage{selnolig}  % disable illegal ligatures
\fi
\IfFileExists{bookmark.sty}{\usepackage{bookmark}}{\usepackage{hyperref}}
\IfFileExists{xurl.sty}{\usepackage{xurl}}{} % add URL line breaks if available
\urlstyle{same} % disable monospaced font for URLs
\hypersetup{
  hidelinks,
  pdfcreator={LaTeX via pandoc}}

\author{}
\date{\vspace{-2.5em}}

\begin{document}

\includegraphics[width=2in,height=1.07014in]{media/image1.png}

UNIVERSIDAD AUTÓNOMA METROPOLITANA

UNIDAD XOCHIMILCO

DIVISIÓN DE CIENCIAS BIOLÓGICAS Y DE LA SALUD

LICENCIATURA EN BIOLOGÍA

\textbf{MÓDULO PRODUCCIÓN PRIMARIA:}

\textbf{Elaboración de la Guía Modular:}

Aldo Aquino Cruz

José A. Arévalo R.

Celia Bulit

Leonor Mendoza Vargas

Aída Marisa Osuna Fernández

Diciembre de 2021

\textbf{CONTENIDO}

\begin{longtable}[]{@{}
  >{\raggedright\arraybackslash}p{(\columnwidth - 2\tabcolsep) * \real{0.8472}}
  >{\raggedright\arraybackslash}p{(\columnwidth - 2\tabcolsep) * \real{0.1250}}@{}}
\toprule()
\endhead
& Página \\
\begin{minipage}[t]{\linewidth}\raggedright
\begin{enumerate}
\def\labelenumi{\arabic{enumi}.}
\item
  Introducción
\item
  Perfil de egreso
\end{enumerate}
\end{minipage} & 2 \\
\begin{minipage}[t]{\linewidth}\raggedright
\begin{enumerate}
\def\labelenumi{\arabic{enumi}.}
\setcounter{enumi}{2}
\tightlist
\item
  Objeto de transformación
\end{enumerate}
\end{minipage} & \\
\begin{minipage}[t]{\linewidth}\raggedright
\begin{enumerate}
\def\labelenumi{\arabic{enumi}.}
\setcounter{enumi}{3}
\tightlist
\item
  Problema eje
\end{enumerate}
\end{minipage} & \\
\begin{minipage}[t]{\linewidth}\raggedright
\begin{enumerate}
\def\labelenumi{\arabic{enumi}.}
\setcounter{enumi}{4}
\tightlist
\item
  Objetivos educativos
\end{enumerate}
\end{minipage} & \\
\begin{minipage}[t]{\linewidth}\raggedright
\begin{enumerate}
\def\labelenumi{\arabic{enumi}.}
\setcounter{enumi}{5}
\item
  Estructura y organización del módulo
\item
  Estrategia didáctica
\end{enumerate}
\end{minipage} & \\
\begin{minipage}[t]{\linewidth}\raggedright
\begin{enumerate}
\def\labelenumi{\arabic{enumi}.}
\setcounter{enumi}{7}
\tightlist
\item
  Investigación modular
\end{enumerate}
\end{minipage} & \\
\begin{minipage}[t]{\linewidth}\raggedright
\begin{enumerate}
\def\labelenumi{\arabic{enumi}.}
\setcounter{enumi}{8}
\tightlist
\item
  Evaluación y acreditación del módulo
\end{enumerate}
\end{minipage} & \\
& \\
\begin{minipage}[t]{\linewidth}\raggedright
\begin{quote}
\textbf{Unidades temáticas:}
\end{quote}

\begin{enumerate}
\def\labelenumi{\Roman{enumi}.}
\item
  PROCESOS ENERGÉTICOS EN LOS ECOSISTEMAS
\item
  PROCESOS METABÓLICOS
\item
  PRODUCTORES PRIMARIOS ACUÁTICOS Y MÉTODOS PARA EVALUAR LA PRODUCCIÓN
  PRIMARIA EN ECOSISTEMAS ACUÁTICOS
\item
  PRODUCTORES PRIMARIOS TERRESTRES Y MÉTODOS PARA EVALUAR LA PRODUCCIÓN
  PRIMARIA EN ECOSISTEMAS TERRESTRES
\end{enumerate}
\end{minipage} & \\
\bottomrule()
\end{longtable}

\textbf{INTRODUCCIÓN}

\textbf{La producción primaria representa una de las funciones del
ecosistema que es clave para la incorporación de energía y la
consecuente generación de biomasa disponible para el resto de la cadena
trófica. La relevancia de este proceso y su diagnóstico, como un
indicador de la estructura y la función en los ecosistemas terrestres y
acuáticos, es el eje teórico-metodológico-práctico de esta Unidad de
Enseñanza-Aprendizaje (UEA). Al ser la producción primaria una función
ecosistémica, su estudio debe ser enfocado desde diversos puntos de
vista y a través de diferentes niveles de organización. Por lo tanto, es
relevante comprender inicialmente los procesos fisicoquímicos y
bioenergéticos que son fundamento para analizar fenómenos y procesos
bioquímicos, metabólicos y fisiológicos que soportan la función
ecosistémica de la producción primaria. Después, se abordan los aspectos
ecofisiológicos y adaptativos de las diversas comunidades de productores
primarios, sus respuestas a las variables ambientales en diversas
escalas espacio-temporales, y los métodos directos e indirectos para su
estudio. Incorporándose los avances sustanciales en la ecología de
ecosistemas es posible entender los procesos ecosistémicos que
contribuyen a la variación global en la estructura y procesos dentro de
los ecosistemas. Describiendo así los mecanismos mediante los cuales los
ecosistemas mantienen su funcionamiento como el flujo de agua y energía
y los ciclos del carbono y nutrientes. Integrando los patrones
temporales y espaciales de los procesos y considerando el efecto
integral de éstos procesos a escala global y sus consecuencias en la
sociedad. Finalmente, la aplicabilidad en la investigación modular,
considerando los servicios ecosistémicos y el uso sustentable de los
productores primarios terrestres y acuáticos.}

\textbf{El perfil deseado del estudiante al ingresar a la UEA incluye un
conocimiento general de los principios de conservación de la materia y
energía, los niveles tróficos en el ecosistema y los ciclos
biogeoquímicos del carbono, nitrógeno, fósforo y azufre, entre otros.
Asimismo, se espera que los estudiantes tengan conocimientos básicos
sobre las moléculas orgánicas y las principales funciones químicas. Los
estudiantes deben tener habilidades para diseñar y realizar muestreos y
registros en campo y laboratorio, análisis y presentación de datos,
incluyendo la aplicación de estadística básica}

\textbf{\hfill\break
}

\textbf{2. PERFIL DE EGRESO DEL MÓDULO.}

\textbf{Al término del módulo, los estudiantes consolidan una formación
que integra conocimiento teórico, metodológico y práctico referente a
los procesos de producción primaria en los ecosistemas acuáticos y
terrestres. Así mismo, desarrollan capacidades reflexivas y comprensivas
acerca de los distintos niveles de funcionamiento de la producción
primaria. Sus competencias profesionales tendrán aplicabilidad en la
investigación y el diagnóstico de la producción primaria vista como un
indicador de la estructura y la función en los ecosistemas terrestres y
acuáticos.}

\textbf{* Obtiene los conocimientos para comprender el proceso de
producción primaria, desde los conceptos fundamentales de quimiosíntesis
y fotosíntesis, hasta las relaciones ecológicas esenciales que lo
promueven.}

\textbf{* Es capaz de aplicar métodos y técnicas de evaluación de este
proceso y de analizar sus resultados en el contexto general de los
flujos energéticos.}

\textbf{* Es capaz de desarrollar estrategias para determinar las
relaciones ecológicas y antropogénicas vinculadas con la producción
primaria en los ambientes terrestres y acuáticos.}

\textbf{* Desarrolla su pensamiento crítico respecto al uso sustentable
de distintos productores primarios, valora la biodiversidad y respeta
los ciclos vitales que aseguren su continuidad.}

\textbf{3. OBJETO DE TRANSFORMACIÓN}

\textbf{``La evaluación de la Producción Primaria''}

\textbf{4. PROBLEMA EJE}

\textbf{¿Cuál es el efecto de las variables ambientales y antrópicas
sobre la incorporación de energía y la acumulación de biomasa en un
ecosistema?}

\textbf{5. OBJETIVOS EDUCATIVOS}\footnote{Los objetivos educativos, los
  contenidos educativos, las estrategias de docencia y los mecanismos de
  evaluación y acreditación corresponden de manera privilegiada a la
  OPERACIÓN modular. Para su selección y formulación es importante
  considerar avances teóricos y metodológicos congruentes con una
  concepción constructivista de la educación, así como el marco
  institucional y las experiencias de trabajo de los integrantes del
  GAD.}

\textbf{Evaluar a la producción primaria como base de la estructura y el
funcionamiento de los ecosistemas y como un elemento para el uso y
conservación de los mismos}

\textbf{6. ESTRUCTURA Y ORGANIZACIÓN DEL MÓDULO}

\textbf{Los contenidos del Módulo Producción Primaria están organizados
en cuatro unidades temáticas distribuidas en 11 semanas de trabajo:}

\begin{enumerate}
\def\labelenumi{\Roman{enumi}.}
\item
  \textbf{Procesos energéticos en los ecosistemas}
\item
  \textbf{Procesos metabólicos}
\item
  \textbf{Productores primarios acuáticos y métodos para evaluar la
  producción primaria en ecosistemas acuáticos}
\item
  \textbf{Productores primarios terrestres y métodos para evaluar la
  producción primaria en ecosistemas terrestres}
\end{enumerate}

\textbf{\hfill\break
}

Unidades

Duración en semanas

1

2

3

4

5

6

7

8

9

10

11

Unidad I

x

x

x

x

x

x

Unidad II

x

x

x

Unidad III

x

x

x

Unidad IV

Investigación modular

x

x

x

x

x

x

x

x

x

x

x

\textbf{7. ESTRATEGIA DIDÁCTICA}

\textbf{Una estrategia didáctica es, en sentido estricto, un
procedimiento organizado, formalizado y orientado a la obtención de una
meta claramente establecida. La estrategia es una guía de acción, que da
sentido y coordinación a todo lo que se hace para llegar a la meta. La
estrategia es flexible y toma forma de acuerdo con las metas a las que
se quiere llegar; hace uso de diversas técnicas, que son procedimientos
didácticos que ayudan a realizar una parte del aprendizaje, e involucran
diferentes actividades más específicas.}

\textbf{En el modelo educativo de la UAM-Xochimilco, las dos
herramientas fundamentales para orientar y conducir el proceso de
enseñanza aprendizaje son: la investigación formativa (investigación
modular) y el trabajo grupal colaborativo. La investigación formativa o
investigación modular es una actividad multidimensional cuya riqueza
``estriba en constituir un proceso integrador de contenidos,
habilidades, actitudes muy diversas, susceptibles de ser enseñados y
aprendidos por una combinación de actividades realizadas en diversos
espacios: aulas, laboratorios, comunidades'' (Ysunza, Bravo, Fernández,
et al, p.47). No produce nuevos conocimientos a nivel social, su novedad
reside en que estimula en el estudiante la construcción/reconstrucción
de conocimientos, habilidades, destrezas y actitudes. La investigación
formativa, en definitiva, favorece un aprendizaje significativo de los
conocimientos validados socialmente. Por otra parte, el sistema modular
defiende la idea de que las relaciones entre las y los estudiantes
inciden de forma decisiva sobre el proceso de socialización en general,
el desarrollo de destrezas y competencias sociales y el rendimiento
escolar. Estas relaciones están estrechamente ligadas a las formas de
organización social de las actividades de aprendizaje y, entre ellas, la
forma de organización cooperativa presenta particularidades de que
favorecen el aprendizaje de todos los estudiantes y contribuye al
desarrollo de sus diversas capacidades.}

\textbf{Para potenciar el alcance y el éxito de estas herramientas
fundamentales: la investigación formativa y el trabajo grupal
colaborativo, es necesario construir una estrategia didáctica plural que
beneficie el aprendizaje de las y los estudiantes, a partir de la
realización de actividades de autoaprendizaje, de aprendizaje
interactivo y de aprendizaje colaborativo}\footnote{Se recomienda la
  lectura de ITESM\ldots..}\textbf{. Fortalecer en el estudiante su
capacidad para el autoaprendizaje es el soporte básico para su desempeño
académico. Las actividades que impulsan el autoaprendizaje son, entre
otras, el estudio individual, la búsqueda, evaluación y análisis de
información de diversa índole, la elaboración de ensayos, fichas,
resúmenes, mapas mentales, mapas conceptuales y otras tareas
individuales. El aprendizaje interactivo se refiere a la realización de
actividades por parte del estudiante, en las que hay algún tipo de
interacción con otra persona, tal como ocurre en el caso de las
exposiciones del profesor, la conferencia de un experto, la realización
de una entrevista, la asistencia a una visita guiada o a un panel, y
otras.}

\textbf{Finalmente, la participación en un grupo de forma colaborativa
tiene efectos favorables en el aprendizaje de los individuos,
dependiendo de los procesos y mecanismos psicológicos, motivacionales y
cognitivos, que se ponen en juego en este tipo de actividades tales como
la solución de problemas, la elaboración de proyectos, el análisis y
discusión de información en grupos, los debates, entre otros. Detrás del
trabajo en equipos colaborativos, subyacen valores de solidaridad, ayuda
mutua y respeto}

\textbf{por las diferencias. Estos valores se expresan a través de
competencias y comportamientos que han de ser desarrollados por los
estudiantes y, por lo tanto, pueden ser enseñados y aprendidos de una
forma intencional y sistemática.}

\textbf{Las estrategias didácticas en el módulo están encaminadas a
proporcionar al estudiante los elementos cognitivos suficientes para
entender el proceso de la producción primaria en ecosistemas acuáticos y
terrestres. También, se espera fortalecer las capacidades para el
trabajo en equipo, el estudio individual y la motivación del uso de
herramientas técnicas, así como, contribuir a la formación profesional
aludiendo a valores como la ética y la responsabilidad, todo ello
estimulando el crecimiento del estudiante, que participe en el curso.}

\textbf{Actividades por equipos:}

\textbf{1. En el aula:}

\textbf{Dinámicas de integración grupal. Exposiciones en equipo de
contenidos específicos de cada unidad utilizando diferentes softwares.
Análisis y discusión de videos con contenido académico relacionado con
el módulo. Elaboración de sistemas de integración de información (mapas
mentales, mapas conceptuales, líneas del tiempo, organigramas, entre
otros).}

\textbf{2. Extra-aula:}

\textbf{Investigación documental en bibliotecas, hemerotecas, bases
electrónicas e Internet. Asistencia a conferencias recomendadas. Visita
a instituciones para entrevistas con especialistas. Asistencia a
talleres y centro de cómputo.}

\textbf{\hfill\break
}

\textbf{3. En laboratorio:}

\textbf{Observación de muestras biológicas. Entrenamiento en técnicas
para evaluar la producción primaria, la biomasa u otros aspectos de las
comunidades de productores primarios. Entrenamiento en el uso de equipos
de registro, de muestreo y de observación.}

\textbf{La selección y planeación de la estrategia didáctica para la
conducción de este módulo, es responsabilidad fundamental del docente
que lo coordina y, para su adecuado desarrollo, demanda el compromiso y
la participación de todos y cada uno de los integrantes del grupo de
estudiantes.}

\hypertarget{investigaciuxf3n-modular-investigaciuxf3n-formativa3.}{%
\section[\textbf{8. INVESTIGACIÓN MODULAR (Investigación
formativa)}\textbf{.}]{\texorpdfstring{\textbf{8. INVESTIGACIÓN MODULAR
(Investigación
formativa)}\footnote{La investigación modular o investigación formativa
  no tiene como propósito la generación de nuevos conocimientos a nivel
  social, sino de \textbf{procesos integradores} de la diversidad de
  saberes que corresponden a cada una de las profesiones que ofrece la
  UAM-X. Se trata de una actividad multidimensional cuya riqueza estriba
  en favorecer en las y los estudiantes, un ``proceso integrador de
  contenidos, habilidades y actitudes muy diversas, susceptibles de ser
  enseñadas y aprendidas por una combinación de actividades realizadas
  en diversos espacios: aulas, laboratorios, comunidades'' (Ysunza,
  Bravo, Fernández, et al, 2019, Hacia la revitalización del sistema
  modular de la UAM Unidad Xochimilco, p.~47). La investigación
  formativa favorece, en cada estudiante, la \textbf{apropiación
  significativa} de los conocimientos socialmente validados.}\textbf{.}}{8. INVESTIGACIÓN MODULAR (Investigación formativa).}}\label{investigaciuxf3n-modular-investigaciuxf3n-formativa3.}}

\textbf{Los contenidos que ofrece este programa se manifiestan de manera
especial durante el desarrollo de la investigación modular, porque
permite integrar los conocimientos adquiridos en el módulo Producción
Primaria, así como enlazarlos con conocimientos alcanzados en los
módulos previos. La investigación modular favorece la práctica de los
métodos y procedimientos propios de las ciencias de la vida y el
desarrollo del pensamiento crítico}\footnote{Un pensador crítico es un
  individuo que está informado y busca información fidedigna; trata de
  identificar los argumentos que sustentan la información que recibe;
  tiene un criterio propio; sabe escuchar; es reflexivo, analiza las
  situaciones, busca alternativas de acción y anticipa las consecuencias
  (Ysunza, Bravo, Fernández, et al, op cit, p.~52).} \textbf{que permite
al estudiante reconocer de manera consciente y significativa las
dimensiones y conexiones presentes en el Problema eje del módulo.}

\textbf{El objetivo general de las investigaciones que las y los
estudiantes realizan durante el trimestre es:}

\textbf{Evaluar a la producción primaria como base de la estructura y el
funcionamiento de los ecosistemas y como un elemento para el uso y
conservación de los mismos.}

\textbf{Así, dependiendo de la trayectoria académica de cada docente
pueden abordarse problemas concretos diversos, por ejemplo:}

\begin{itemize}
\item
  \textbf{Evaluación de la Producción Primaria en ecosistemas
  terrestres.}
\item
  \textbf{Evaluación de la Producción Primaria en ecosistemas acuático.}
\item
  \textbf{Evaluación de las tasas de crecimientos de especies vegetales
  bajo condiciones limitantes.}
\item
  \textbf{Evaluación de la Producción Primaria en sistemas de producción
  acuícola.}
\item
  \textbf{Evaluación de la biomasa en pie de masas forestales}
\item
  \textbf{Aplicación del Meta-\/-análisis en experimentos de
  crecimientos y manejo de la producción primaria.}
\end{itemize}

\textbf{El papel del docente en la investigación formativa consiste en
diseñar, programar, dirigir y orientar, de manera cercana, las tareas
necesarias y pertinentes que realizarán las y los estudiantes en equipos
de trabajo colaborativo, para lograr esa integración de saberes y
aprendizajes propios del ejercicio de la profesión del biólogo, así como
de habilidades que son comunes a las distintas formaciones
universitarias, y que están estrechamente vinculadas con la
investigación científica en sentido más estricto}\footnote{Entre estas
  habilidades comunes destacan: habilidades perceptivas, instrumentales
  (ej. desarrollo de operaciones cognitivas: inferencia, análisis,
  síntesis, interpretación; manejo del lenguaje formal para leer,
  escribir, hablar y escuchar; aprender a observar y a preguntar);
  habilidades de pensamiento lógico; habilidades de construcción
  conceptual y de construcción metodológica; habilidades metacognitivas
  y habilidades para la construcción social del conocimiento:
  colaboración, comunicación, socialización (Cfr. Ysunza, Bravo,
  Fernández, et al, op. cit. pp.~47-48)}\textbf{.}

\textbf{La investigación formativa favorece especialmente el aprendizaje
de la terminología especializada en la jerga profesional y en las
teorías que la sustentan, así como el manejo apropiado de métodos,
técnicas, equipos y herramientas propias del ejercicio profesional del
biólogo.}

\textbf{Dependiendo del tema propuesto por las/los docentes, para
vincular la investigación formativa con el objeto de transformación del
módulo se solicita a los estudiantes:}

\begin{itemize}
\item
  \textbf{Revisión bibliográfica actualizada de documentos arbitrados
  (consultando fuentes nacionales e internacionales).}
\item
  \textbf{Elaboración del protocolo de investigación y diseño de
  instrumentos de recolección de información.}
\item
  \textbf{Aplicación de instrumentos de recolección de información.}
\item
  \textbf{Procesamiento y análisis de datos.}
\item
  \textbf{Elaboración de un informe escrito.}
\item
  \textbf{Presentación del informe final, destacando el impacto social y
  las perspectivas para el desarrollo de nuevas investigaciones que
  ayuden a proporcionar elementos adicionales para resolver el problema
  eje.}
\end{itemize}

\textbf{El informe escrito debe reunir ciertos requisitos para
garantizar su calidad: presentación y organización adecuadas; claridad
en el uso de indicadores y conceptos; redacción ágil y buena ortografía;
cuadros y gráficas correctamente elaborados; bibliografía pertinente,
actualizada y adecuadamente referida. Asimismo, el informe incluirá un
apartado de observaciones o recomendaciones que el equipo considere
pertinentes y valiosas para ser entregadas a la institución o la
comunidad en la que fue realizada la investigación modular.}

\textbf{9. EVALUACIÓN Y ACREDITACIÓN DEL MÓDULO}

\textbf{La evaluación es vista habitualmente como sinónimo de
calificación y de enjuiciamiento ``objetivo'' y terminal de las
capacidades y el aprovechamiento de las y los estudiantes. Esta es una
concepción limitada, que vale la pena transformar hacia una orientación
de la evaluación como instrumento de intervención para mejorar el
proceso de enseñanza y aprendizaje en la licenciatura en Biología de la
UAM-X. El sentido de la evaluación se acerca más a la comprensión del
proceso educativo que tiene lugar en las aulas y se aleja de la
intención medidora o sancionadora que a menudo se asocia a este proceso,
confundiendo la evaluación con el mero acto de calificar.}

\textbf{Si en este modelo educativo, el/la docente guía y coordina el
proceso educativo, es corresponsable de los resultados que se obtienen
en el módulo, por lo que le interesa saber qué apoyos requieren las y
los estudiantes para avanzar y alcanzar los logros que pretende el
programa. Si en el sistema modular la investigación formativa y el
trabajo grupal colaborativo son las principales herramientas pedagógicas
para favorecer el aprendizaje, es preciso mantener un seguimiento y una
retroalimentación constante que oriente e impulse el trabajo cotidiano.
``Esto es lo que ocurre en los equipos de investigación que funcionan
correctamente y eso es lo que tiene sentido también en una situación de
aprendizaje orientada a la construcción de conocimientos y (apoyada en)
la investigación (modular)'' (Sánchez, Gil y Martínez, 1996, p.~17).
Este seguimiento y esta retroalimentación continua y enriquecedora, es a
lo que se refiere la evaluación formativa, que se convierte así en un
instrumento de apoyo para el aprendizaje y para la mejora de la
enseñanza.}

\textbf{En síntesis, el proceso de evaluación es permanente y consiste
en dar seguimiento y retroalimentación a las actividades de aula, campo,
laboratorio y gabinete que realizan los alumnos. Su función principal es
favorecer el aprendizaje. No obstante, toda evaluación posee una
dimensión valorativa, es decir, la asignación dentro de una escala
determinada de un valor, el cual es referido como referente,
``calificador'', que permite al estudiante percibir que tan lejos o
cerca se encuentra su desempeño de un nivel satisfactorio. Así, la
calificación aparece como complemento de la evaluación formativa, y
manifiesta la estimación de los logros de cada estudiante. Esta
estimación ha de ir acompañada de propuestas que el/la docente ofrece
para que el/la estudiante progrese. Además, se sustenta en ``evidencias
académicas tangibles'' con las que el/la estudiante muestra su progreso
en el logro de los objetivos educativos del módulo.}

\textbf{En el módulo ``Producción Primaria'', las evidencias académicas
que mostrará el estudiante, así como el peso que cada una de ellas tiene
en la estimación de la calificación, se muestran en el siguiente
cuadro:}

\begin{quote}
\textbf{Participación en clase 10\%}

\textbf{Entrega de trabajos (clase y laboratorio) 20\%}

\textbf{Evaluaciones objetivas de contenidos teóricos 40\%}

\textbf{Investigación modular 30\%}
\end{quote}

\textbf{La evaluación debe considerar el aprendizaje de contenidos en
relación con los objetivos específicos. Esta parte se evaluará mediante
evaluaciones (exámenes) individuales por unidad o unidades temáticas.
Opcionalmente, la evaluación de contenidos se puede complementar con la
preparación de seminarios de investigación documental de ciertas
unidades. El promedio del puntaje de las evaluaciones (exámenes) de
contenidos valdrá un 50\% de la evaluación y tendrá que obtener una
calificación aprobatoria, mínimo 6, para acreditar la UEA. El restante
50 \% se refiere a la adquisición de habilidades y al cumplimiento de
las actividades de gabinete, laboratorio y campo incluidas en la
investigación modular, aunque estas actividades son por equipo, se hará
una réplica individual para asegurar que cada miembro del equipo
participó activamente.}

\textbf{Evaluación diagnóstica}

\textbf{Al inicio del trimestre, se aplicará un cuestionario diagnóstico
basado en los prerrequisitos que el alumno deberá tener al iniciar la
UEA, considerando la seriación de los módulos. Si la evaluación revela
que los alumnos traen información insuficiente, se recomendarán lecturas
de regularización.}

\textbf{\hfill\break
UNIDAD TEMÁTICA I}

\textbf{1.- PROCESOS ENERGÉTICOS EN LOS ECOSISTEMAS}

\textbf{Introducción}

\textbf{La vida y todos los procesos biológicos en la biosfera requieren
energía. La unidad describe los conceptos elementales de energía
electromagnética y bioenergética, desde el nivel atómico hasta los
flujos energéticos que ocurren en la biosfera. Se resaltan los
mecanismos fundamentales de incorporación de energía y los procesos de
transferencia energética que impulsan la dinámica de la producción
primaria. Un concepto fundamental para entender los procesos biológicos
y la vida misma es el de la energía. La aplicación de las leyes de la
termodinámica como parte de la explicación y definición de la vida,
presenta todo un reto en la interpretación de la dinámica y procesos
ecológicos. En tanto que la transformación de la materia inorgánica y
energía para la generación de biomasa establece el punto de partida en
cualquier red trófica.}

\textbf{Objetivo General}

\textbf{Analizar el proceso de incorporación de energía al ecosistema y
discutir los conceptos básicos relacionados con la producción}

\textbf{primaria}

\textbf{Objetivos específicos}

\begin{itemize}
\item
  \textbf{Identificar y definir la importancia del nivel de organización
  de los productores primarios}
\item
  \textbf{Comprender los términos sobre los flujos de energía y
  producción primaria}
\item
  \textbf{Revisar la aplicación de los procesos termodinámicos que
  influyen en la producción primaria}
\end{itemize}

\textbf{Preguntas clave}

\begin{enumerate}
\def\labelenumi{\arabic{enumi}.}
\item
  \textbf{¿Cuál es la importancia de los productores primarios en el
  ecosistema?}
\item
  \textbf{¿Qué conceptos de termodinámica que se pueden aplicar en la
  descripción de las relaciones ecológicas que dan origen a la
  Producción Primaria?}
\item
  \textbf{¿Cómo se aplican los conceptos de flujo de energía para
  describir la estructura y función que presentan los ecosistemas?}
\item
  \textbf{¿Cuáles son las determinantes del intercambio total de energía
  en los ecosistemas?}
\item
  \textbf{¿La absorción de energía por el ecosistema de ve influenciada
  por cuáles propiedades climáticas y ecosistémicas?}
\item
  \textbf{¿Cuál es el impacto de la sociedad humana en la producción
  primaria de los ecosistemas?}
\end{enumerate}

\textbf{Contenidos educativos}

\textbf{1.1 FLUJOS DE ENERGÍA Y PRODUCCIÓN PRIMARIA}

\begin{itemize}
\item
  \textbf{Definición de Ecología y niveles de organización de los seres
  \textgreater{} vivos con énfasis en el concepto de ecosistemas.}
\item
  \textbf{Distinción entre procesos y funciones del ecosistema: a)
  \textgreater{} fotosíntesis b) producción primaria}
\item
  \textbf{El papel de la producción primaria en las redes y pirámides
  \textgreater{} tróficas en los ecosistemas.}
\item
  \textbf{Estructura y limitantes del ecosistema, número de Reynolds,
  \textgreater{} controles de los procesos en el ecosistema, Sucesión
  ecológica, \textgreater{} dinámicas de regulación interna en el
  ecosistema, \textgreater{} retroalimentación, estado estable.}
\item
  \textbf{Pensamiento sistémico, Concepto de modelos, ejemplos y usos.}
\end{itemize}

\textbf{1.2 TERMODINÁMICA}

\begin{itemize}
\item
  \textbf{Concepto y tipos de energía: Energía cinética, potencial y
  libre de Gibbs.}
\item
  \textbf{Leyes de la termodinámica: 0, 1ª, 2ª, 3ª; Procesos
  termodinámicos: adiabático, isocórico, isobárico e isotérmico.}
\item
  \textbf{Plantas endotérmicas y ejemplos de animales homeotermos,
  poiquilotermos, endotérmicos y exotérmicos.}
\item
  \textbf{Conceptos y unidades de medida de la producción primaria:
  productividad primaria, producción primaria bruta (PPB) y neta (PPN),
  biomasa, cosecha en pie, eficiencia fotosintética, respiración,
  controles ambientales de la PPN, Asignación de la PPN, ciclos diurnos
  y estacionales de la asignación de la PPN, ciclos circadianos,
  fotoperíodo y fenología.}
\item
  \textbf{Diferencias de los biomas en la PPN, efectos de los disturbios
  y clima en la PPN.}
\item
  \textbf{Impactos humanos en los ecosistemas, servicios ambientales,
  sustentabilidad y apropiación de la PPN.}
\end{itemize}

\textbf{Actividades}

\textbf{Se propone que individualmente y/o en grupos de trabajo las y
los estudiantes:}

\begin{itemize}
\item
  \textbf{Revisen publicaciones recientes sobre el estudio de la
  aplicación del pensamiento sistémico en la ecología de ecosistemas.}
\item
  \textbf{Localicen y presenten artículos recientes sobre el impacto de
  la sociedad humana en los ecosistemas con énfasis en el manejo,
  servicios ambientales y apropiación de la naturaleza.}
\item
  \textbf{Analicen artículos de investigación sobre las diferentes
  formas de estimación de la productividad primaria.}
\end{itemize}

\textbf{Duración:}

\textbf{2 semanas}

\textbf{Bibliografía Básica}

\textbf{Blanco JA. ¿Por qué necesitamos modelos ecológicos en la gestión
de recursos naturales? En: Aplicaciones de modelos ecológicos a la
gestión de recursos naturales. España: OmniaScience; 2013}

\textbf{Blanco JA. Modelos ecológicos: descripción, explicación y
predicción. Ecosistemas. 2013; 22(3).}

\textbf{Chapin, F.S., III, G.P. Kofinas, and C. Folke. Principles of
Ecosystem Stewardship: Resilience- Based Natural Resource Management in
a Changing World. New York. USA; Springer; 2009}

\textbf{Chapin III, F. S., Matson, P. A., y Vitousek, P. Principles of
terrestrial ecosystem ecology. New York. USA; Springer Science \&
Business Media; 2011}

\textbf{Çengel YA, Boles, MA. Termodinámica, 7ª Edición. México; Mac
Graw Hill; 2012.}

\textbf{Covich, A. P. Energy flow and ecosystems. En: Levin, editor,
Encyclopedia of Biodiversity, vol.~3, N.Y. USA; Academic
Press--Elsesvier Inc 2013; p.~237--249.}

\textbf{Ellis E.C., and N. Ramankutty. Putting people on the map:
Anthropogenic biomes of the world. Frontiers in Ecology and the
Environment 2008; 6:439--447.}

\textbf{Golley, F.B. A History of the Ecosystem Concept in Ecology: More
than the Sum of the Parts. New Haven, USA; Yale University Press; 1993}

\textbf{Haberl, H., Erb, K. H., Krausmann, F., Gaube, V., Bondeau, A.,
Plutzar, C., Gingrich, S., Lucht, W., y Fischer-Kowalski, M. Quantifying
and mapping the human appropriation of net primary produc- tion in
earth's terrestrial ecosystems. Proceedings of the National Academy of
Sciences, 2007; 104(31);12942--12947.}

\textbf{Hart, R. D. Conceptos Básicos sobre agroecosistemas. Turrialba,
Costa Rica.; Centro Agronómico Tropical de Investigación y Enseñanza;
1985}

\textbf{Jørgensen, S. E. y Fath, B. D. Application of thermodynamic
principles in ecology. Ecological Complexity, 2004; 1(4):267--280.}

\textbf{Kalff, J. Limnology. Upper Saddle River, NJ. USA; Prentice-Hall;
2002}

\textbf{Kareiva, P. y Marvier, M. Ecology, concepts and theories in. En
Levin 2013, editor, Encyclopedia of Biodiversity, volumen 3, pp.~1--8.
N.Y. USA: Academic Press--Elsevier Inc., 2nd ed.; 2013}

\textbf{Kerkhoff, A.J., B.J. Enquist, J.J. Elser, and W.F. Fagan. Plant
allometry, stoichiometry and the temperature-dependence of primary
productivity. Global Ecology and Biogeography. 2005; 14:585--598.}

\textbf{Liévano M., F. y Londoño, J. E. El pensamiento sistémico como
heramienta metodológica para la resolución de problemas. Soluciones de
Posgrado EIA, 2012; (8):43--65.}

\textbf{Madigan MT, Martinko JM, Bender KS, Buckley DH, Stahl DA. Brock
Biology of Microorganisms; 14ª edición. Science Progress. 2015}

\textbf{Margalef, R. Teoría de los sistemas ecológicos. D.F. México;
Alfaomega Grupo Editor, S.A. de C.V.,2 ed.; 2002}

\textbf{Mann, K.H. and J.R.N. Lazier. Dynamics of Marine Ecosystems:
Biological-Physical Interactions in the Oceans. 3a ed.~Victoria,
Australia; Blackwell Publishing; 2006}

\textbf{MEA (Millennium Ecosystem Assessment). Ecosystems and Human
Well-being: Synthesis. Washington DC. USA, Island Press; 2005.}

\textbf{Monson, R. y Baldocchi, D. Terrestrial Biosphere-Atmosphere
fluxes. Cambridge, UK.; Cambridge University Press; 2014}

\textbf{Odum, E. P. Ecosystem, concept of. En Levin, S. A., editor,
Encyclopedia of Biodiversity, volumen 3, pp.~59--63. Levin2013,
Amsterdam, The Netherlands; Academic Press-\/-Elsevier Inc.~2 ed.; 2013}

\textbf{Odum, H. T. Environment, power and society for the twenty-first
century. The Hierarchy of Energy. New York, USA; Columbia University
Press; 2007}

\textbf{Roy, J., Saugier, B., y Mooney, H. A., ed.~Terrestrial global
productivity. California, USA; Academic Press; 2001}

\textbf{Roxburgh, S. H., Berry, S. L., Buckley, T. N., Barnes, B., y
Roderick, M. L. What is NPP? Inconsistent accounting of respiratory
fluxes in the definition of net primary production. Functional Ecology.
2005; 19(3), 378-382.}

\textbf{Scheiner, S. M. y Willing, M. R. A general theory of ecology.
Theor. Ecol. 2007; (1):21--28.}

\textbf{Schlesinger, W.H. Biogeochemistry: An Analysis of Global Change.
San Diego. C., USA; Academic Press; 1997}

\textbf{Virginia, R. A. y Wall, D. H. Ecosystems function, principles
of. En Levin, S. E., editor, Encyclopedia of Biodiversity, volumen 3.
Amsterdam, The Netherlands; Academic Press--Elsevier Inc.; 2013}

\textbf{Waring, R.H. and S.W. Running. Forest Ecosystems: Analysis at
Multiple Scales. 3rd ed.~San Diego. C. USA; Academic Press; 2007}

\textbf{\hfill\break
}

\textbf{Bibliografía Complementaria}

\textbf{Anderson, T. R., Boersma, M., y Raubenheimer, D. Stoichiometry:
Linking elements to biochemicals. Ecology, 2004; 85(5):1193--1202.}

\textbf{Beerling, D. Quantitative estimates of changes in marine and
terrestrial primary productivity over the past 300 million years.
Proceedings of the Royal Society of London. Series B: Biological
Sciences,1999; 266(1431):1821--1827.}

\textbf{Bernacchi, C. J., Bagley, J. E., Serbin, S. P., Ruiz-Vera, U.
M., Rosenthal, D. M., y Vanloocke, A. Modelling C3 photosynthesis from
the chloroplast to the ecosystem. Plant, Cell \& Environment, 2013;
36(9):1641--1657.}

\textbf{Castañares M., E. J. Sistemas Complejos y Gestión Ambiental: el
caso del corredor Biol ógico Mesoamericano Mexico. Número 6 en Serie
Conocimientos. CDMX, México; Comisión Nacional para el Conocimineto y
Uso de la Biodiversidad; 2009}

\textbf{Cleveland, C. C., Houlton, B. Z., Smith, W. K., Marklein, A. R.,
Reed, S. C., Parton, W., Del Grosso, S. J., y Running, S. W. Patterns of
new versus recycled primary production in the terrestrial biosphere.
Proceedings of the National Academy of Sciences, 2013;
110(31):12733--12737.}

\textbf{DeLucia, E. H., Drake, J. E., Thomas, R. B., y Gonzalez-Meler,
M. Forest carbon use efficiency: is respiration a constant fraction of
gross primary production? Global Change Biology, 2007;
13(6):1157--1167.}

\textbf{Hessen, D. O. Too much energy? Ecology, 2004; 85(5):1177--1178.}

\textbf{Jaffe RL, Taylor W. Biological Energy. En: The Physics of
Energy; 2019}

\textbf{Landsberg, J. J. y Waring, R. H. A generalised model of forest
productivity using simplified concepts of radiation--use efficiency,
carbon balance and partitioning. Forest Ecology and Management, 1997;
95:209--228.}

\textbf{Marone, L. y Bunge, M. La explicación en ecología. Boletín de la
Asociación Argentina de Ecología, 1998; 7(2):35--37.}

\textbf{Mota-Vargas C, Encarnación-Luévano A, Ortega-Andrade HM,
Prieto-Torres DA, Peña-Peniche A, Rojas-Soto OR. Una breve introducción
a los modelos de nicho ecológico. In: La biodiversidad en un mundo
cambiante: Fundamentos teóricos y metodológicos para su estudio. 2019.}

\textbf{Schulze, E.-D., Beck, E., Buchmann, N., Clemens, S.,
Müller-Hohenstein, K., y Michael, S.-L. Plant Ecology. Berlin,
Heidelberg, Germany; Springer--Verlag, 2 ed.; 2019}

\textbf{Sinsabaugh, R. L. y Follstad Shah, J. J. Ecoenzymatic
stoichiometry and ecological theory. Annual Review of Ecology,
Evolution, and Systematics, 2012; 43(1):313--343}

\textbf{\hfill\break
}

\textbf{UNIDAD TEMÁTICA II}

\textbf{PROCESOS METABÓLICOS}

\textbf{Introducción}

\textbf{La formación de moléculas orgánicas en la base de la trama
trófica y el subsecuente crecimiento de los productores primarios, son
resultado de dos procesos biológicos fundamentales en la biosfera: la
fotosíntesis y la quimiosíntesis. Los mecanismos metabólicos,
bioquímicos y fisiológicos involucrados en la fotosíntesis y la
quimiosíntesis. El crecimiento de los productores primarios depende de
su capacidad para incorporar el dióxido de carbono (CO2) en compuestos
de carbono orgánico, por medio de la utilización de la energía luminosa
y/o y energía química, durante los procesos de fotosíntesis y
quimiosíntesis. Estos procesos involucran diferentes reacciones químicas
y procesos físicos, además de las diferencias estructurales
(fisiológicas y morfológicas) de cada uno de los organismos primarios.
La comprensión de los mecanismos metabólicos, de los ciclos de
transformación de la materia y la energía al interior de la célula, así
como las modalidades que han alcanzado estos mecanismos a lo largo de la
evolución, son un referente indispensable para entender posteriormente
los aspectos ecológicos de la producción primaria.}

\textbf{En esta unidad analizarán las diferencias químicas, fisiológicas
y morfológicas de los productores primarios que realizan la
quimiosíntesis, fotosíntesis y respiración.}

\textbf{Objetivo General}

\textbf{Comprender y conceptualizar los procesos bioquímicos y
fisiológicos, desde el nivel subatómico hasta el ecológico, que
sustentan el proceso de la producción primaria.}

\textbf{Objetivos específicos}

\begin{itemize}
\item
  \textbf{Determinar los principales mecanismos de incorporación de
  energía en organismos autótrofos}
\item
  \textbf{Comprender los procesos bioquímicos, metabólicos, celulares y
  fisiológicos que realizan los productores primarios.}
\item
  \textbf{Analizar la relación de los procesos metabólicos de los
  productores primarios en los ecosistemas.}
\end{itemize}

\textbf{Preguntas clave}

\begin{enumerate}
\def\labelenumi{\arabic{enumi}.}
\item
  \textbf{¿Cuáles son los mecanismos que utilizan los organismos
  autótrofos en la incorporación de energía?}
\item
  \textbf{¿Cuáles son los factores limitantes de la fotosíntesis,
  ambientales, ecológicos y fisiológicos?}
\item
  \textbf{¿Cuáles son las diferencias fundamentales entre las vías
  fotosintéticas C3, C4 y CAM?}
\item
  \textbf{¿Qué papel juegan los organismos fotosintéticos en el flujo de
  energía dentro del ecosistema?}
\item
  \textbf{¿Cuál es la importancia de los modelos fotosintéticos de las
  plantas C3 y C4?}
\item
  \textbf{¿Cuáles son las diferentes metodologías que existen para el
  estudio de la fotosíntesis?}
\item
  \textbf{¿Cuáles son las diferencias principales entre las vías
  fotosintéticas C3, C4 y CAM?}
\end{enumerate}

\textbf{Contenidos educativos}

\textbf{2.1 QUIMIOSINTESIS}

\begin{itemize}
\tightlist
\item
  \textbf{Quimiosíntesis. Concepto, proceso en bacterias sulfurosas,
  férricas y metanógenas Rutas energéticas, importancia ecológica y
  grupos de organismos que las presentan}
\end{itemize}

\textbf{2.2. FASE FOTOQUÍMIICA DE LA FOTOSINTESIS Y SU REGULACIÓN}

\begin{itemize}
\item
  \textbf{Naturaleza de la luz. Espectro electromagnético. Fracción
  fotosintéticamente activa de la luz. Coeficiente de extinción de la
  luz.}
\item
  \textbf{Fenómenos luminiscentes: fluorescencia, fosforescencia,
  quimioluminiscencia, bioluminiscencia}
\item
  \textbf{Historia de la investigación de la fotosíntesis}
\item
  \textbf{Estructura y evolución del cloroplasto y sistemas
  pigmentarios.}
\item
  \textbf{Fotosíntesis an\uline{oxigénica}}
\item
  \textbf{Fotosíntesis oxigénica bacteriana fotosistema 870}
\item
  \textbf{Fotosíntesis oxigénica en las plantas: fotosistema 680 y 730;
  complejo antena (procesos fotofísicos) y centro de reacción (procesos
  fotoquímicos); cambio de energía solar a energía química
  (fotofosforilación cíclica y acíclica).}
\item
  \textbf{El fenómeno de la fotoinhibición del aparato fotosintético;
  Disminución de absorción de luz (movimiento algas, cutículas gruesas,
  orientación foliar, movimientos cloroplásticos, pigmentos no
  fotosintéticos en epidermis); Ciclo de las xantofilas; Inactivación y
  reparación cíclica en el psi (polipéptido d132kda-psii).}
\item
  \textbf{Mixotrofía}
\end{itemize}

\textbf{2.3. FASE BIOQUÍMICA DE LA FOTOSINTESIS Y SU REGULACIÓN
(Fijación del carbono y nitrógeno)}

\begin{itemize}
\item
  \textbf{Vías de incorporación del carbono en plantas C3: ciclo de
  Calvin-Benson}
\item
  \textbf{Ruta metabólica C2 o fotorrespiración.}
\item
  \textbf{Vías de incorporación del carbono en plantas C4 (ruta de
  Hatch-Slack)}
\item
  \textbf{Intermediarios C3-C4}
\item
  \textbf{Vías de incorporación del carbono en plantas CAM (metabolismo
  ácido de las crasuláceas)}
\item
  \textbf{Organismos incompletos y facultativos CAM}
\item
  \textbf{Mecanismos Fotosintéticos de las plantas acuáticas}
\item
  \textbf{Plantas CAM acuáticas}
\item
  \textbf{Balance energético de la fotosíntesis}
\item
  \textbf{Conductividad estomatal y capa límite}
\item
  \textbf{Isótopos estables, uso eficiente del agua y uso eficiente del
  nitrógeno}
\item
  \textbf{Factores internos reguladores de la fotosíntesis, metabólicos
  y fisiológicos}
\item
  \textbf{Diferencias fisiológicas, bioquímicas y anatómicas entre
  plantas tolerantes e intolerantes a la sombra}
\item
  \textbf{Respuesta de la fotosíntesis a la radiación variable,
  activación, inducción post-iluminación, ganancia neta de carbono y
  crecimiento}
\item
  \textbf{Disponibilidad de agua, regulación estomatal, relación del uso
  eficiente del agua, transpiración, asimilación y concentración de
  carbono interno}
\item
  \textbf{Estrés hídrico y fraccionamiento de isótopos de carbono}
\item
  \textbf{Fluorescencia de la clorofila, relación con el rendimiento
  fotosintético}
\item
  \textbf{Productos de la fotosíntesis y su regulación por
  retroalimentación}
\item
  \textbf{Temperatura de la hoja y fotosíntesis, efectos y adaptaciones}
\item
  \textbf{Mecanismos asociados con la asimilación de carbono en la
  fotosíntesis de las plantas acuáticas}
\item
  \textbf{Fijación de nitrógeno en cianobacterias}
\item
  \textbf{Nitrificación en suelo}
\item
  \textbf{Vías de incorporación del nitrógeno: N\textsubscript{2},
  NO\textsubscript{3}, NO\textsubscript{2} en las plantas}
\item
  \textbf{Fijación de N\textsubscript{2} atmosférico en la industria de
  los fertilizantes}
\end{itemize}

\textbf{2.4 RESPIRACIÓN}

\begin{itemize}
\item
  \textbf{Mitocondrial: vía alternativa (resistencia al cianuro) y su
  relación con la termogénesis en las plantas.}
\item
  \textbf{Radical: relación con el intercambio catiónico en el suelo y
  la nutrición vegetal: concepto de factor limitante.}
\item
  \textbf{Teoría de evapotranspiración: absorción del agua y solutos.}
\end{itemize}

\textbf{Actividades}

\textbf{Se propone que individualmente y/o en grupos de trabajo las y
los estudiantes:}

\begin{itemize}
\item
  \textbf{Revisen publicaciones recientes sobre el estudio de la
  fotosíntesis.}
\item
  \textbf{Localicen y presenten artículos recientes sobre el uso de
  diferentes técnicas para estimar las tasas de fotosíntesis.}
\item
  \textbf{Analicen artículos de investigación sobre las respuestas
  ecofisiológicas de los organismos fotosintéticos al cambio climático.}
\item
  \textbf{Realicen observaciones en campo o en laboratorio sobre el
  efecto de los factores ambientales en la fotosíntesis.}
\end{itemize}

\textbf{Duración:}

\textbf{4 semanas}

\textbf{Bibliografía Básica}

\textbf{Azcón-Bieto, J. y Talón, M., ediores. Fundamentos de Fisiología
vegetal. McGraw-Hill Interamericana Edicons Universitat de Barcelona,
Madrid, España; 2000}

\textbf{Baker NR. Chlorophyll fluorescence: A probe of photosynthesis in
vivo. Vol. 59, Annual Review of Plant Biology. 2008.}

\textbf{Book Review: Brock Biology of Microorganisms -- 14th edition.
Science Progress. 2016;99(3).}

\textbf{Boynton WR, Kemp WM. Chapter 18: Estuaries. Nitrogen in the
Marine Environment. 2008.}

\textbf{Bräutigam A, Gowik U. Photorespiration connects C3 and C4
photosynthesis. Vol. 67, Journal of Experimental Botany. 2016}

\textbf{Campbell PKE, Huemmrich KF, Middleton EM, Ward LA, Julitta T,
Daughtry CST, et al.~Diurnal and seasonal variations in chlorophyll
fluorescence associated with photosynthesis at leaf and canopy scales.
Remote Sensing. 2019; 11(5).}

\textbf{Chapin III, F. S., Matson, P. A., y Vitousek, P. Principles of
terrestrial ecosystem ecology. N.Y. USA; Springer Science \& Business
Media; 2011}

\textbf{Evert, R. F. y Eichhorn, S. E. Raven Biology of Plants. N.Y.
USA; W. H. Freeman and Company Publishers; 2013}

\textbf{Falkowski PG, Katz ME, Knoll AH, Quigg A, Raven JA, Schofield O,
et al.~The evolution of modern eukaryotic phytoplankton. Science. 2004;
vol.~305.}

\textbf{Hammer AC, Pitchford JW. The role of mixotrophy in plankton
bloom dynamics, and the consequences for productivity. ICES Journal of
Marine Science. 2005; 62(5).}

\textbf{INTAGRI. Plantas C3, C4 Y CAM. Serie Nutrición Vegetal. 2018;
125(1).}

\textbf{Jameson, D. M. Introduction to Fluorescence. Boca Raton, FL,
USA; CRC Press; 2014}

\textbf{Lambers, H., F.S. Chapin, III, and T.L. Pons. Plant
Physiological Ecology. 2a ed.~New York. USA; Springer; 2008}

\textbf{Leininger, S., Urich, T., Schloter, M., Schwark, L., Qi, J.,
Nicol, G. W., Prosser, J. I., Schuster, S. C., y Schleper, C. Archaea
predominate among ammonia-oxidizing prokaryotes in soils. Nature, 2006;
442(442):806--809.}

\textbf{Lüttge, U. Physiological Ecology of Tropical Plants. Berlin,
Germany; Springer--Verlag; 1997}

\textbf{Monson, R. y Baldocchi, D. Terrestrial Biosphere-Atmosphere
fluxes. Cambridge, UK; Cambridge University Press; 2014}

\textbf{Noble, P. S. Physicochemical and Environmetal Plant Physiology.
4a ed. Oxford, UK; Academic Press; 2009}

\textbf{Reynolds CS. The ecology of phytoplankton. The Ecology of
Phytoplankton; 2006.}

\textbf{Roy, J., Saugier, B., y Mooney, H. A., ed.~Terrestrial global
productivity. California, USA; Academic Press; 2001}

\textbf{Sage, R.F. The evolution of C4 photosynthesis. New Phytologist
2004; 161:341--370.}

\textbf{Santhanam P, Begum A, Pachiappan P. Basic and applied
Phytoplankton biology. Basic and Applied Phytoplankton Biology. 2018.}

\textbf{Schulze, E.-D., Beck, E., Buchmann, N., Clemens, S.,
Müller-Hohenstein, K., y Michael, S.-L. Plant Ecology. 2a ed.Berlin,
Heidelberg, Germany; Springer--Verlag; 2019}

\textbf{Solomon EP, Martin CE, Martin DW, Berg LR. Biology. 11th.
ed.~Boston; Cengage Learning; 2019}

\textbf{Sotelo A. Fotosíntesis. Facultad de Ciencias Exactas y Naturales
y Agrimensura; 2014}

\textbf{Stoecker DK, Hansen PJ, Caron DA, Mitra A. Mixotrophy in the
Marine Plankton. Annual Review of Marine Science. 2017; 9(1).}

\textbf{Strous, M., Fuerst, J. A., Kramer, E. H., Logemann, S., Muyzer,
G., van de Pas-Schoonen, K. T., Webb, R., Kuenen, J. G., y Jetten, M. S.
M. Missing lithotroph identified as new planctomycete. Nature 1999;
400(6743):446--449}

\textbf{Ward BA, Follows MJ. Marine mixotrophy increases trophic
transfer efficiency, mean organism size, and vertical carbon flux.
Proceedings of the National Academy of Sciences of the United States of
America. 2016; 113(11)}

\textbf{Waring, R.H. and S.W. Running. Forest Ecosystems: Analysis at
Multiple Scales. 3a ed.~San Diego. C. USA; Academic Press; 2007}

\textbf{Young SNR, Sack L, Sporck-Koehler MJ, Lundgren MR. Why is C4
photosynthesis so rare in trees? Vol. 71, Journal of Experimental
Botany; 2020.}

\textbf{Bibliografía Complementaria}

\textbf{Berry, J.A. \& Björkman, O. Photosynthetic response and
adaptation to temperature in higher plants. Annu. Rev.~Plant Physiol.
1980; 31: 491---543}

\textbf{Bowes, G. \& Salvucci, M.E. Plasticity in the photosynthetic
carbon metabolism of submersed aquatic macrophytes. Aquat. Bot. 1989;
34: 233---286}

\textbf{Brown, R.H. \& Hattersley, P.W. Leaf anatomy of C3---C4 species
as related to evolution of C4 photosynthesis. Plant Physiol. 1989;
91:1543---1550}

\textbf{Boyer, J.S. Water transport. Annu. Rev.~Plant Physiol. 1985;
36:473---516}

\textbf{Bunce, J.A. Carbon dioxide effects on stomatal responses to the
environment and water use by crops under field conditions. Oecologia
2004; 140:1---10}

\textbf{Chapin III, F.S. Effects of plant traits on ecosystem and
regional processes: A conceptual framework for predicting the
consequences of global change. Ann. Bot. 2003; 91: 455---463}

\textbf{Chazdon, R.L. y Pearcy, R.W. The importance of sunflecks for
forest understory plants.BioSciences 1991; 41: 760---766}

\textbf{Evans, J.R. y Loreto, F. Acquisition and diffusion of
CO\textsubscript{2} in higher plant leaves. In: Photosynthesis:
physiology and metabolism, R.C. Leegood, T.D. Sharkey, \& S. Von
Caemmerer (eds.). Dordrecht, Holanda; Kluwer Academic Publishers; 2000
p.~321---351}

\textbf{Evans, J.R., Sharkey, T.D., Berry, J.A., y Farquhar, G.D. Carbon
isotope discrimination measured with gas exchange to investigate CO2
diffusion in leaves of higher plants. Aust. J. Plant Physiol. 1986;
13:281---292}

\textbf{Falkowski, P., Scholes, R.J., Boyle, E., Canadell, J., Canfield,
D., Elser, J., Gruber, N., Hibbard, K., Högberg, P., Linder, S.,
Mackenzie, F.T., Moore III, B., Pedersen, T., Rosenthal, Y., Seitzinger,
S., Smetacek, V., y Steffen, W. The global carbon cycle: a test of our
knowledge of Earth as a system. Science 2000; 290:291---296}

\textbf{Farquhar, G.D. \& Richards, R.A. Isotopic composition of plant
carbon correlates with water-use efficiency of wheat genotypes. Aust. J.
Plant Physiol. 1984; 11:539---552}

\textbf{Farquhar, G.D., Von Caemmerer, S., \& Berry, J.A. A biochemical
model of photosynthetic CO\textsubscript{2} assimilation in leaves of C3
species. Planta 1980; 149: 78---90}

\textbf{Farquhar, G.D., O'Leary, M.H., \& Berry, J.A. On the
relationship between carbon isotope discrimination and the intercellular
carbon dioxide concentration in leaves. Aust. J. Plant Physiol. 1982;
9:131---137.}

\textbf{Field, C.B., Ball, T., \& Berry, J.A. Photosynthesis: principles
and field techniques. In: Plant physiological ecology; field methods and
instrumentation, R.W. Pearcy, J.R. Ehleringer, H.A. Mooney, \& P.W.
Rundel (eds.). London, UK; Chapman and Hall; 1989 p.~209---253.}

\textbf{Flexas, J., Diaz-Espejo, A., Berry, J.A., Cifre, J., Galmés, J.,
Kaidenhoff, R., Medrano, H. \& Ribas-Carbo, M. Analysis of leakage in
IRGA's leaf chambers of open gas exchange systems: quantification and
its effects in photosynthesis parameterization. J. Exp. Bot. 2007;
58:1533---1543.}

\textbf{Flexas, J., Ribas-Carbó, M., Diaz-Espejo, A., Galmés, J., \&
Medrano, H. Mesophyll conductance to CO\textsubscript{2}: current
knowledge and future prospects. Plant Cell Environ. 2008; 31:602---621.}

\textbf{Field, C.B., Lobell, D.B., Peters, H.A., \& Chiariello, N.R.
Feedbacks of terrestrial ecosystems to climate change. Annu. Rev.~Env.
Res. 2007; 32:1---29.}

\textbf{Foley, J.A., Costa, M.H., Delire, C., Ramankutty, N., \& Snyder,
P. Green surprise? How terrestrial ecosys- tems could affect earth's
climate. Front. Ecol. Environ.2003; 1:38---44.}

\textbf{Grace, J. Understanding and managing the global carbon cycle. J.
Ecol. 2004; 92:189---202.}

\textbf{Hatch, M.D.~\& Slack, C.R. 1966. Photosynthesis by sugar cane
leaves A new carboxylation reaction and the pathway of sugar formation.
Biochem. J 1966; 101:103---111.}

\textbf{Law, B.E., E. Falge, L. Gu, D.D. Baldocchi, P. Bakwin et al.
Environmental controls over carbon dioxide and water vapor exchange of
terrestrial vegetation. Agricultural and Forest Meteorology 2002;
113:97--120.}

\textbf{Long, S.P. y Hällgren, J.E. Measurement of CO\textsubscript{2}
assimilation by plants in the field and the laboratory. In:
Photosynthesis and production in a changing environment, D.O. Hall,
J.M.O. Scurlock, H.R. Bolhàr-Nor- denkampf, R.C. Leegood, \& S.P. Long
(eds.). London, UK; Chapman and Hall; 1993 p.~129---167.}

\textbf{Long, S.P., Ainsworth, E.A., Rogers, A., \& Ort, D.R. Rising
atmospheric carbon dioxide: plants FACE the future. Annu. Rev.~Plant
Biol. 2004; 55: 591---628.}

\textbf{Monsi, M. \& Saeki T. On the factor light in plant communities
and its importance for matter production. Ann. Bot. 2005; 95:549---567.}

\textbf{Pearsall, W.H. The soil complex in relation to plant
communities. J. Ecol. 1938; 26:180---193.}

\textbf{Von Caemmerer, S. A model of photosynthetic CO\textsubscript{2}
assimilation and carbon-isotope discrimination in leaves of certain
C3---C4 intermediates. Planta 1989; 178: 463---474.}

\textbf{Von Caemmerer, S. Biochemical models of leaf photo- synthesis.
Collingwood, Australia; CSIRO Publishing; 2000}

\textbf{Von Caemmerer, S. \& Farquhar, G.D. Some relation- ships between
biochemistry of photosynthesis and gas exchange of leaves. Planta 1981;
153:376---387}

\textbf{\hfill\break
}

\textbf{UNIDAD TEMÁTICA III}

\textbf{Productores primarios acuáticos / Métodos para evaluar la
Producción Primaria en ecosistemas acuáticos}

\textbf{Introducción}

\textbf{La producción primaria de sistemas acuáticos es regulada por
diversas variables ambientales (bióticas y abióticas). En esta Unidad se
analiza el efecto de distintas variables independientes que controlan la
producción primaria en ambientes acuáticos. En este sentido se prioriza
la comprensión de las variables físicas, químicas y biológicas que
determinan la variabilidad de la producción primaria en los ecosistemas
acuáticos.}

\textbf{Para comprender e interrelacionar los procesos metabólicos que
subyacen a la productividad primaria, es indispensable abordar el
conocimiento de los diferentes grupos de productores primarios en los
ecosistemas acuáticos, sus características principales, respuestas
adaptativas y evolutivas y, patrones espacio-temporales de distribución
así como el papel que juegan en los diversos ecosistemas. Esta unidad se
enfoca en los principales biomas marinos y dulceacuícolas de México. Las
variables bióticas, abióticas y antrópicas, tienen un efecto fundamental
sobre la vida de los productores primarios. Para comprender cómo la
variabilidad de los parámetros de tipo físico, químico y biológico, en
el tiempo y el espacio, inciden en una mayor o menor producción
primaria, es importante conocer los métodos, equipos, unidades de medida
y sus equivalencias, utilizados para su registro. Desde las técnicas in
vitro hasta las estimaciones con sensores remotos en ecosistemas
acuáticos para su uso y conservación en relación con la producción
primaria y algunos ejemplos sobresalientes que caracterizan a este
fenómeno a diferentes escalas}

\textbf{Objetivo General}

\textbf{Desarrollar la capacidad de comprensión e integración de los
conceptos fundamentales aplicados al estudio de la producción primaria,
desde el nivel celular hasta el ecosistémico, de los distintos métodos
empleados en la evaluación de la producción primaria. Señalando sus
ventajas y desventajas, con el propósito de evaluar y valorar la función
ecológica de diversas variables implicadas en la variabilidad y la
complejidad de las comunidades de productores primarios de ambientes
acuáticos de la biósfera. Asimismo, el alumno será capaz de hacer un
análisis de los efectos implicados en la estructura y función de los
ecosistemas acuáticos a nivel de los productores primarios.}

\textbf{Objetivos específicos}

\begin{itemize}
\item
  \textbf{Identificar y caracterizar a los diferentes productores
  primarios acuáticos y las comunidades que integran.}
\item
  \textbf{Comprender las variables bióticas, abióticas y antrópicas, que
  tienen efecto sobre los niveles de producción primaria.}
\item
  \textbf{Revisar y comprender los fundamentos, ventajas y desventajas
  de los métodos y técnicas para evaluar la producción primaria en
  ecosistemas acuáticos.}
\item
  \textbf{Conocer los servicios ecosistémicos que brindan los
  productores primarios acuáticos y las posibles medidas de mitigación
  en los ecosistemas que habitan, sobre los problemas antropogénicos
  relacionados con el cambio climático.}
\end{itemize}

\textbf{\hfill\break
}

\textbf{Preguntas clave}

\begin{enumerate}
\def\labelenumi{\arabic{enumi}.}
\item
  \textbf{¿Cuáles son las características biológicas de los productores
  primarios acuáticos?}
\item
  \textbf{¿Por qué las variables bióticas, abióticas y antrópicas
  modifican los niveles de producción primaria acuática?}
\item
  \textbf{¿Cuáles son las ventajas y desventajas de los métodos y
  técnicas para evaluar la producción primaria en ecosistemas
  acuáticos?}
\item
  \textbf{¿Cuáles son las posibles medidas de mitigación sobre los
  problemas antropogénicos en los ecosistemas acuáticos?}
\item
  \textbf{¿Cómo los productores primarios nos brindan servicios
  ecosistémicos?}
\end{enumerate}

\textbf{Contenidos educativos}

\textbf{3. PRODUCTORES PRIMARIOS ACUÁTICOS}

\textbf{3.1 UNICELULARES Y PLURICELULARES}

\begin{itemize}
\item
  \textbf{Fitoplancton: características morfológicas generales, tipos de
  pigmentos, reproducción, hábitat, importancia ecológica
  (florecimientos nocivos) de: cianobacterias, cocolitofóridos,
  silicoflagelados, dinoflagelados, diatomeas y clorofitas.}
\item
  \textbf{Microfitobentos, macroalgas bénticas y pleustónica, pastos
  marinos y manglares: principales integrantes, patrones de
  distribución, adaptaciones morfológicas y fisiológicas, papel en el
  ecosistema, comunidades asociadas.}
\item
  \textbf{Servicios ecosistémicos, problemas antropogénicos y medidas
  posibles de mitigación.}
\end{itemize}

\textbf{3.2 VARIABLES QUE REGULAN LA PRODUCCIÓN PRIMARIA}

\begin{itemize}
\item
  \textbf{Luz: calidad, intensidad y propiedades. modelo
  fotosíntesis-intensidad luminosa. Fitoplancton de sol y de sombra.
  Zona eufótica, afótica y disfótica. Profundidad de compensación.
  Distribución vertical de la biomasa y de la producción.}
\item
  \textbf{Nutrientes: distribución de nutrientes en la columna de agua
  (capa de mezcla). Relación C:N y C:P. Pastoreo. Control ``bottom-up''
  y ``top-dow''. Eutrofización.}
\item
  \textbf{Temperatura y salinidad: termoclina y haloclina.}
\item
  \textbf{Turbulencia y movimientos de las masas de agua: olas, mareas,
  corrientes de arrastre, giros, frentes, surgencias (efecto del cambio
  climático global, El Niño y La Niña).}
\item
  \textbf{Sistemas modificados por el ser humano: fertilización de
  estanques, cultivos extensivos e intensivos.}
\item
  \textbf{Producción primaria en ecosistemas acuáticos (concepto,
  origen, clasificación y adaptación): ventilas hidrotermales, arrecifes
  coralinos, lagunas costeras, lagos, estuarios, mares y océanos.}
\end{itemize}

\begin{quote}
\textbf{3.3 MÉTODOS DIRECTOS E INDIRECTOS PARA EVALUAR LA PRODUCCIÓN
PRIMARIA (fundamentos, ventajas y desventajas, y equipos)}
\end{quote}

\begin{itemize}
\item
  \textbf{Biomasa del fitoplancton: unidades de medición y métodos para
  su estimación. Contenido de pigmentos: espectrofotometría,
  fluorometría, sensores remotos (CZCS, SeaWIFS, MODIS).}
\item
  \textbf{Producción primaria en el fitoplancton: unidades de medición y
  métodos para su estimación. Conceptos e interpretación: producción
  primaria, productividad primaria, producción primaria bruta y neta,
  producción nueva y regenerada. Cambios en la concentración de oxígeno
  (botellas DBO, Demanda Biológica de Oxígeno), incorporación de carbono
  radioactivo.}
\item
  \textbf{Métodos cuantitativos y cualitativos para el estudio de la
  estructura de la comunidad fitoplanctónicas : cuantificación y
  biovolumen; método de Utermöhl, cámaras (Palmer-Maloney,
  Sedgwick-Rafter, Neubauer, Petroff-Hauser, sedimentación), citometría
  de flujo, inmunofluorescencia y epifluorescencia}
\item
  \textbf{Métodos para evaluar la biomasa y la producción primaria en
  macrófitas acuáticas (pastos marinos y macroalgas): biomasa en pie,
  marcado, intervalo del plastocrón, cambios en la concentración de
  oxígeno}
\end{itemize}

\textbf{Actividades}

\textbf{Se propone que individualmente y/o en grupos de trabajo las y
los estudiantes:}

\begin{quote}
\textbf{● Analicen literatura especializada para determinar la
clasificación y la diversidad de los principales grupos de productores
primarios en ecosistemas acuáticos.}

\textbf{● Elaboraren e interpreten gráficos utilizando programas
estadísticos para determinar las interrelaciones entre distintas
variables dependientes (bióticas) y variables independientes
(abióticas).}

\textbf{● A través de la búsqueda y el análisis de literatura
especializada, compararan y contrastaran las ventajas y desventajas de
los principales métodos y técnicas aplicadas al estudio de la producción
primaria en ecosistemas acuáticos.}

\textbf{● Empleen el conocimiento teórico en su trabajo de investigación
para determinar el funcionamiento basal de los productores primarios en
los sistemas acuáticos de estudio.}

\textbf{● Realicen diversas actividades prácticas instrumentales, en
campo y laboratorio, para desarrollar sus capacidades de investigación
científica.}

\textbf{● Realicen experimentos,~\emph{in vitro}~e~\emph{in vivo}, para
comprender el funcionamiento de poblaciones de productores primarios y
las variables que los regulan en distintos tipos de ambientes.}

\textbf{● Obtengan habilidades para el manejo de diferentes instrumentos
ópticos para estudiar a las comunidades de productores primarios en
ecosistemas acuáticos.}

\textbf{● Evalúen el uso, las aplicaciones y los alcances de diversas
estrategias de manejo de las comunidades de los productores primarios en
ecosistemas acuáticos.}
\end{quote}

\textbf{Duración:}

\textbf{3 semanas}

\textbf{Bibliografía Básica}

\textbf{Azcón, J., Fleck, I., Aranda, X., and Goméz N. Fotosintesis,
factores ambientales y cambio climático. In: Fundamentos de Fisiología
Vegetal; 2008.}

\textbf{Barreiro, M. T. y Signoret, M. Productividad primaria en
sistemas acuáticos costeros. Métodos de evaluación. D.F. México;
UAM-Xochimilco; 1999}

\textbf{Baslam M, Mitsui T, Hodges M, Priesack E, Herritt MT, Aranjuelo
I, et al.~Photosynthesis in a Changing Global Climate: Scaling Up and
Scaling Down in Crops. Vol. 11, Frontiers in Plant Science; 2020.}

\textbf{Beerling, D. Quantitative estimates of changes in marine and
terrestrial primary productivity over the past 300 million years.
Proceedings of the Royal Society of London. Series B: Biological
Sciences, 1999; 266(1431)1821--1827.}

\textbf{Bhatla SC, A. Lal M. Plant Physiology, Development and
Metabolism. Plant Physiology, Development and Metabolism; 2018}

\textbf{Carpenter JH. Determination of dissolved oxygen by Winkler
Titration. Environmental Chemistry of Boston Harbor 2006; (1888)}

\textbf{Castro P, Huber ME. Chapter 14 Coral Reefs In Marine Biology.
Marine Biology; 2010}

\textbf{Chakraborty S, Nielsen LT, Andersen KH. Trophic strategies of
unicellular plankton. American Naturalist. 2017; 189(4).}

\textbf{Church MJ, Cullen JJ, Karl DM. Approaches to measuring marine
primary production. In: Encyclopedia of Ocean Sciences; 2019}

\textbf{Clark DA, Brown S, Kicklighter DW, Chambers JQ, Thomlinson JR,
Ni J. Measuring net primary production in forests: Concepts and field
methods. Ecological Applications. 2001; 11(2).}

\textbf{Das S, Bhattacharya SS. Environmental stress and stress biology
in plants. In: Plant Secondary Metabolites, Volume Three: Their Roles in
Stress Eco-physiology; 2017}

\textbf{Deborah A. Clark, Sandra Brown, David W. Kicklighter, Jeffrey Q.
Chambers, John R. Thomlinson Ajn, Clark Da, Brown S, Kicklighter DW,
Chambers JQ, Thomlinson JR, et al.~Measuring net primary production in
forests: concepts and field methods. Ecological Applications. 2001;
11(2).}

\textbf{Dodds WK, Whiles MR. Freshwater ecology. Freshwater Ecology:
Concepts and Environmental Applications of Limnology; 2019}

\textbf{Fahey TJ, Knapp AK. Principles and Standards for Measuring
Primary Production. Principles and Standards for Measuring Primary
Production; 2007}

\textbf{Field CB, Behrenfeld MJ, Randerson JT, Falkowski P. Primary
production of the biosphere: Integrating terrestrial and oceanic
components. Science. 1998; 281(5374)}

\textbf{Gazeau F, Middelburg JJ, Loijens M, Vanderborght JP, Pizay MD,
Gattuso JP. Planktonic primary production in estuaries: Comparison of
\textsuperscript{14}C, O\textsubscript{2} and 18O methods. Aquatic
Microbial Ecology. 2007; 46(1)}

\textbf{Geider RJ, Moore CM, Suggett DJ. Ecology of marine
phytoplankton. In: Ecology and the Environment. 2014}

\textbf{Hogarth PJ. The Biology of Mangroves and Seagrasses. The Biology
of Mangroves and Seagrasses. 2015}

\textbf{Jayathilake DRM, Costello MJ. Seagrass Biome. In: Encyclopedia
of the World's Biomes. 2020}

\textbf{Jeffrey SW, Humphrey GF. New spectrophotometric equations for
determining chlorophylls a, b, c and c2 in higher plants, algae and
natural phytoplankton, Biochemie und Physiologie der Pflanzen, 1975; p
191-194.}

\textbf{Keeling PJ. The endosymbiotic origin, diversification and fate
of plastids. Vol. 365, Philosophical Transactions of the Royal Society
B: Biological Sciences. 2010}

\textbf{Körner C. Plant Adaptations to Alpine Environments. In:
Encyclopedia of the World's Biomes. 2020}

\textbf{Lamalakshmi Devi E, Kumar S, Basanta Singh T, Sharma SK,
Beemrote A, Devi CP, et al.~Adaptation strategies and defence mechanisms
of plants during environmental stress. In: Medicinal Plants and
Environmental Challenges. 2017}

\textbf{Lauenroth WK. Methods of Estimating Belowground Net Primary
Production. In: Methods in Ecosystem Science. 2000}

\textbf{Macedo MF, Duarte P, Ferreira JG. The influence of incubation
periods on photosynthesis-irradiance curves. Journal of Experimental
Marine Biology and Ecology. 2002; 274(2).}

\textbf{Marcolla B, Migliavacca M, Rödenbeck C, Cescatti A. Patterns and
trends of the dominant environmental controls of net biome productivity.
Biogeosciences. 2020; 17(8).}

\textbf{Pace ML, Lovett GM. Primary production: The foundation of
ecosystems. In: Fundamentals of Ecosystem Science. 2012.}

\textbf{Pinet PR. Invitation to Oceanography. Jones and Bartlett
Publishers. 2009.}

\textbf{Poikane S, Kelly M, Cantonati M. Benthic algal assessment of
ecological status in European lakes and rivers: Challenges and
opportunities. Vol. 568, Science of the Total Environment. 2016.}

\textbf{Reißer W. Algae---Anatomy, Biochemistry, and Biotechnology.
Journal Of Plant Physiology. 2007;164(3).}

\textbf{Restrepo S. Cuantificación de Clorofila ``\emph{a}'' Clorofilas.
Departamento de Química Orgánica, UNAM. 2017.}

\textbf{Ryu Y, Berry JA, Baldocchi DD. What is global photosynthesis?
History, uncertainties and opportunities. Remote Sensing of Environment.
2019;223.}

\textbf{Sims, D. A., Rahman, A. F., Cordova, V. D., El‐Masri, B. Z.,
Baldocchi, D. D., Flanagan, L. B., y Xu, L. On the use of MODIS EVI to
assess gross primary productivity of North American ecosystems. Journal
of Geophysical Research: Biogeosciences (2005--2012), 111(G4); 2006}

\textbf{Singh RP, Reddy CRK. Unraveling the functions of the macroalgal
microbiome. Frontiers in Microbiology. 2016; 6(JAN)}

\textbf{Siobhan Fennessy M, Cronk JK. Primary production and
respiration: Ecological processes in Wetlands. In: The Wetland Book: I:
Structure and Function, Management, and Methods. 2018.}

\textbf{Smith NG, Keenan TF, Colin Prentice I, Wang H, Wright IJ,
Niinemets Ü, et al.~Global photosynthetic capacity is optimized to the
environment. Vol. 22, Ecology Letters. 2019.}

\textbf{Staehr PA, Testa JM, Kemp WM, Cole JJ, Sand-Jensen K, Smith S.
v. The metabolism of aquatic ecosystems: History, applications, and
future challenges. Vol. 74, Aquatic Sciences. 2012}

\textbf{Taíz and Zeiger. Photosynthesis: Physiological and Ecological
Considerations. Plant physiology. 2006}

\textbf{Valk AG van der. Wetlands: Classification. In: Wetlands and
Habitats; 2020}

\textbf{Westberry TK, Behrenfeld MJ. Oceanic Net Primary Production. In
2014}

\textbf{Wood AM, Everroad RC, Wingard LM. Measuring growth rates in
microalgal cultures. En: Andersen RA, Editor. Algal culturing
techniques. London; Elsevier Academic Press; 2005. p 269-326}

\textbf{\hfill\break
}

\textbf{UNIDAD TEMÁTICA IV}

\textbf{Productores primarios terrestres / Métodos para evaluar la
Producción Primaria en ecosistemas terrestres}

\textbf{Introducción}

\textbf{La producción primaria en los ecosistemas terrestres es regulada
por diversas variables ambientales (bióticas y abióticas). En esta
Unidad se analizan los efectos de distintas variables independientes que
controlan la producción primaria en diversos ecosistemas terrestres.
Particularmente, se prioriza la comprensión de las variables físicas,
químicas y biológicas que fundamentalmente determinan la variabilidad de
la producción primaria en ambientes terrestres. Para comprender e
interrelacionar los procesos metabólicos que subyacen a la productividad
primaria, es indispensable abordar el conocimiento de los grupos de
productores primarios en los ecosistemas, sus características
principales, adaptaciones y patrones espacio-temporales de distribución
así como el papel que juegan en los diversos ecosistemas.}

\textbf{Esta unidad se enfoca hacia las principales comunidades que se
encuentran en los biomas terrestres de México. Las variables bióticas,
abióticas y antrópicas, tienen un efecto fundamental sobre la vida de
los productores primarios. Para comprender como la variabilidad de los
parámetros de tipo físico, químico y biológico, en el tiempo y el
espacio, inciden en una mayor o menor producción primaria, es importante
conocer los métodos, equipos, unidades de medida y sus equivalencias,
utilizados para su registro. Desde las técnicas in vitro hasta las
estimaciones con sensores remotos en ecosistemas terrestres para su uso
y conservación en relación a la producción primaria y algunos ejemplos
sobresalientes que caracterizan a este fenómeno a diferentes escalas}

\textbf{\hfill\break
}

\textbf{Objetivo General}

\textbf{Desarrollar la capacidad de comprensión e integración de los
conceptos fundamentales aplicados al estudio de la producción primaria,
desde el nivel celular hasta el ecosistémico, de los distintos métodos
empleados en la evaluación de la producción primaria. Señalando sus
ventajas y desventajas, con el propósito de evaluar y valorar la función
ecológica de diversas variables implicadas en la variabilidad y la
complejidad de las comunidades de productores primarios de ambientes
terrestres de la biósfera. Asimismo, el alumno será capaz de hacer un
análisis de los efectos implicados en la estructura y función de los
ecosistemas terrestres a nivel de los productores primarios.}

\textbf{Objetivos específicos}

\begin{itemize}
\tightlist
\item
  \textbf{Identificar y caracterizar a los diferentes productores
  primarios terrestres y las comunidades que integran.}
\end{itemize}

\begin{itemize}
\item
  \textbf{Comprender las variables bióticas, abióticas y antrópicas, que
  tienen efecto sobre los productores primarios terrestres.}
\item
  \textbf{Revisar y comprender los fundamentos, ventajas y desventajas
  de los métodos y técnicas para evaluar la producción primaria en
  ecosistemas terrestres.}
\item
  \textbf{Conocer los servicios ecosistémicos que brindan los
  productores primarios terrestres y las posibles medidas de mitigación
  en los ecosistemas que habitan, sobre los problemas antropogénicos}
\end{itemize}

\textbf{\hfill\break
}

\textbf{Preguntas clave}

\begin{enumerate}
\def\labelenumi{\arabic{enumi}.}
\item
  \textbf{¿Cuáles son las características biológicas de los productores
  primarios terrestres?}
\item
  \textbf{¿Por qué las variables bióticas, abióticas y antrópicas
  modifican los niveles de producción primaria terrestre?}
\item
  \textbf{¿Cuáles son las ventajas y desventajas de los métodos y
  técnicas para evaluar la producción primaria en ecosistemas
  terrestres?}
\item
  \textbf{¿Cuáles son las posibles medidas de mitigación sobre los
  problemas antropogénicos en los ecosistemas terrestres?}
\item
  \textbf{¿Cómo los productores primarios nos brindan servicios
  ecosistémicos?}
\end{enumerate}

\textbf{Contenidos educativos}

\textbf{4. PRODUCTORES PRIMARIOS TERRESTRES}

\textbf{4.1 BIOMAS TERRESTRES}

\begin{itemize}
\item
  \textbf{Patrones de distribución en México y en el mundo,
  características morfofisiológicas de un espécimen típico del grupo
  dominante en un ecosistema determinado}
\item
  \textbf{Importancia ecológica y adaptaciones morfofisiológicas de los
  productores primarios en cada tipo de bioma: tundra; bosques
  templados; praderas; desiertos; sabanas; palmares, bosques tropicales}
\item
  \textbf{Producción primaria en ecosistemas terrestres (concepto,
  origen, clasificación y adaptación morfológicas y fisiológicas):
  tundra; bosques templados; praderas; desiertos; sabanas; palmares,
  bosques tropicales; humedales}
\item
  \textbf{Sistemas modificados por el ser humano: cultivos extensivos e
  intensivos, uso de agua y fertilizantes en sistemas agrícolas}
\item
  \textbf{Cambio climático, calentamiento global, medidas de mitigación
  y producción primaria en: tundra; bosques templados; praderas;
  desiertos; sabanas; palmares, bosques tropicales; humedales}
\item
  \textbf{Servicios ecosistémicos de los biomas terrestres}
\end{itemize}

\textbf{4.2 VARIABLES QUE REGULAN LA PRODUCCIÓN PRIMARIA}

\begin{itemize}
\item
  \textbf{Respuestas al ambiente, estrés, resistencia, evasión y
  tolerancia. Respuestas al estrés, aclimatación y adaptación}
\item
  \textbf{Radiación activa fotosintética, Medición fotoperiodo e
  intensidad}
\item
  \textbf{Respuestas fotosintéticas de las plantas tolerantes a la
  sombra, características estructurales Bioquímicas y de intercambio de
  gases}
\item
  \textbf{Regímenes hídricos (efectos de la sequía, humedad ambiental y
  precipitación pluvial, déficit de vapor de agua)}
\item
  \textbf{Efectos del exceso de irradiancia}
\item
  \textbf{Efecto de los nutrientes en el suelo, Relación nitrógeno
  fotosíntesis (Uso eficiente del nitrógeno fotosintético)}
\item
  \textbf{Relación entre uso eficiente de agua, del nitrógeno
  fotosintético, transpiración y fotosíntesis}
\item
  \textbf{Interacciones Nitrógeno Luz y agua en la fotosíntesis}
\item
  \textbf{Efectos de la temperatura, salinidad y pH}
\item
  \textbf{Efectos de los contaminantes aéreos en la fotosíntesis}
\item
  \textbf{Efecto de la concentración de CO2 en la atmósfera y sus
  consecuencias en la fotosíntesis, crecimiento y balance de carbono en
  las plantas}
\end{itemize}

\textbf{4.3 MÉTODOS DIRECTOS E INDIRECTOS PARA EVALUAR LA PRODUCCIÓN
PRIMARIA TERRESTRE (fundamentos, ventajas y desventajas, y equipos)}

\begin{itemize}
\item
  \textbf{Contenido de pigmentos, concentración de bióxido de carbono,
  cuantificación de biomasa, teledetección, muestreos y censos.}
\item
  \textbf{Métodos para evaluar la biomasa y la producción primaria en
  vegetación terrestre y ecotonal: determinación del índice de área
  foliar, biomasa foliar y de madera, desarrollo de ecuaciones
  alométricas, análisis dimensional, trampas de hojarasca, intercambio
  de gases, fluorescencia}
\item
  \textbf{Métodos para el estudio cualitativo y cuantitativo de
  vegetación terrestre y ecotonal: censos de la comunidad, biomasa en
  pie, muestreo, cuadrantes, uso de sensores remotos: escalamiento,
  imágenes de satélite, fotografías y mapas}
\item
  \textbf{Contenido de pigmentos, concentración de bióxido de carbono,
  teledetección, muestreos y censos}
\item
  \textbf{Análisis de crecimiento clásico y funcional}
\item
  \textbf{Cavitación, curvas de vulnerabilidad, raíces, tallo y peciolo}
\item
  \textbf{Medición de intercambio de gases, curvas de respuesta a la luz
  y al CO2, Modelo de fotosíntesis (Farquhar)}
\item
  \textbf{Fraccionamiento de isótopos de carbono en relación con el uso
  eficiente del agua}
\item
  \textbf{Covarianza de remolinos (turbulencias) métodos
  microclimáticos}
\item
  \textbf{Isótopos estables de Carbono}
\item
  \textbf{Circulación del agua en las plantas, medidas de flujo de
  savia, disipación de calor, pulso y balance}
\item
  \textbf{Conductividad hidráulica suelo y hojas, dosel, madera y
  peciolo. Parámetros hidráulicos presión-volúmen. Teoría de la
  cohesión-tensión}
\item
  \textbf{Ecuaciones alométricas y análisis dimensional}
\end{itemize}

\textbf{Actividades}

\textbf{Se propone que individualmente y/o en grupos de trabajo las y
los estudiantes:}

\begin{itemize}
\item
  \textbf{Revisen publicaciones recientes sobre el estudio de la
  producción primaria.}
\item
  \textbf{Localicen y presenten artículos recientes sobre el uso de
  diferentes técnicas para estimar la productividad primaria.}
\item
  \textbf{Analicen artículos de investigación sobre las respuestas
  ecofisiológicas de los organismos a las variables climáticas.}
\item
  \textbf{Realicen observaciones en campo o en laboratorio sobre el
  efecto de los factores ambientales en la producción primaria.}
\item
  \textbf{Evaluaran las tasas de crecimiento bajo diferentes factores
  ambientales, como nutrientes, luz, humedad temperatura.}
\end{itemize}

\textbf{Duración:}

\textbf{3 semanas}

\textbf{Bibliografía Básica}

\textbf{Adams, B., Andrew White, and T. M. Lenton. An analysis of some
diverse approaches to modelling terrestrial net primary productivity.
Ecological Modelling 2004; 177(3-4)353-391}

\textbf{Amo R. del, S. y Vergara Tenorio, M. del C., ed.~La Restauración
Ecológica productiva. El camino para recuperar el patrimonio biocultural
de los pueblos Mesoamericanos. Xalapa, Ver. México; Universidad
Veracruzana; 2019}

\textbf{Beerling, D. Quantitative estimates of changes in marine and
terrestrial primary productivity over the past 300 million years.
Proceedings of the Royal Society of London. Series B: Biological
Sciences, 1999; 266(1431):1821--1827}

\textbf{Causton, D. R. y Venus, J. C. The Biometry of Plant Growth.
London, UK.; Arnold Publishers; 1981}

\textbf{Chapin III, F. S., Matson, P. A., y Vitousek, P. Principles of
terrestrial ecosystem ecology. N.Y. USA; Springer Science \& Business
Media.; 2011}

\textbf{DeFries, R. Remote sensing and image processing. En Levin, S.
A., editor, Encyclopedia of Biodiversity, volumen 6, pp.~389--399.
Amsterdam, The Netherlands.; Academic Press--Elsevier Inc.; 2013}

\textbf{DeLucia, E. H., Drake, J. E., Thomas, R. B., y Gonzalez-Meler,
M. Forest carbon use efficiency: is respiration a constant fraction of
gross primary production? Global Change Biology, 2007;
13(6):1157--1167.}

\textbf{Fahey, T. J. y Knapp, A. K., ed.~Principles and Standars for
Measuring Primary Production. N.Y., USA.; Oxford University Press; 2007}

\textbf{Haberl, H., Erb, K. H., Krausmann, F., Gaube, V., Bondeau, A.,
Plutzar, C., Gingrich, S., Lucht, W., y Fischer-Kowalski, M. Quantifying
and mapping the human appropriation of net primary production in earth's
terrestrial ecosystems. Proceedings of the National Academy of Sciences,
2007; 104(31):12942-- 12947.}

\textbf{Hessen, D. O., Agren, G. I., Anderson, T. R., Elser, J. J., y de
Ruiter, P. C. Carbon sequestration in ecosystems: The role of
stoichiometry. Ecology, 2004; 85(5):1179--1192.}

\textbf{Hunt, R. Plant Growth Analysis. Studies in Biology No.~96.
London, U. K.; Edward Arnold; 1978}

\textbf{Hunt, R. Plant Growth Curves: The functional approach to plant
growth analysis. London, U. K.; Edward Arnold;1982}

\textbf{Hunt, R. Basic Growth Analysis: Plant growth analysis for
beginners. London, UK.; Unwin Hyman Ltd; 1990}

\textbf{Hunt, R., Causton, D. R., Shipley, B., y Askew, A. P. A modern
tool for classical plant growth analysis. Annals of Botany, 2002;
90:485--488.}

\textbf{Kloeppel, B. D., Harmon, M. E., y Fahey, T. J. Principles and
standards for measuring primary production, Cap. Estimating above ground
net primary productivity in forest-dominated ecosystems, pp.~63--81. New
York, N.Y., USA.; Oxford University Press; 2007}

\textbf{Lambers, H., Chapin III, F. S., y Pons, T. L. Plant
Physiological Ecology. New York, USA.; Springer Verlag, 2a ed.; 2008}

\textbf{Landsberg, J. J. y Gower, S. T. Applications of physiological
ecology to forest management. Physiological ecology series. San Diego,
Cal. U.S.A.; Academic Press; 1997}

\textbf{Landsberg, J. J. y Waring, R. H. A generalised model of forest
productivity using simplified concepts of radiation--use efficiency,
carbon balance and partitioning. Forest Ecology and Management, 1997;
95:209--228.}

\textbf{Masera, O. R., Bellon, M. R., y Segura, G. Forest management
options for sequestering carbon in Mexico. Biomass and Bioenergy, 1995;
8(5):357--367.}

\textbf{Masera, O. R., Bellon, M. R., y Segura, G. Forest management
options for sequestering carbon in Mexico. Biomass and Bioenergy, 1995;
8(5):357--367.}

\textbf{Niklas, K. J. Plant Allometry. The scaling of form and process.
Chicago, I. U.S.A.; The University of Chicago Press; 1994}

\textbf{Roy, J., Saugier, B., y Mooney, H. A., ed.~Terrestrial global
productivity. California, USA.; Academic Press; 2001}

\textbf{Schulze, E.-D., Beck, E., Buchmann, N., Clemens, S.,
Müller-Hohenstein, K., y Michael, S.-L. (2019). Plant Ecology. Berlin,
Heidelberg, Germany; Springer--Verlag 2 ed.; 2019}

\textbf{Vieira, D. L. M., Holl, K. D., y Peneireiro, F. M.
Agro-successional restoration as a strategy to facilitate tropical
forest recovery. Restoration Ecology, 2009; 17(4):451--459.}

\textbf{\hfill\break
}

\textbf{Bibliografía Complementaria}

\textbf{Gómez-Pompa, A. y Kaus, A. Traditional management of tropical
forests in Mexico. En Anderson, A. B., editor, Alternatives to
deforestation: steps toward sustainable use of the Amazon Rain Forest,
pp.~45--64. N.Y., U.S.A.; Columbia University Press; 1990}

\textbf{Jørgensen, S. E. y Fath, B. D. Application of thermodynamic
principles in ecology. Ecological Complexity, 2004; 1(4):267--280.}

\end{document}
